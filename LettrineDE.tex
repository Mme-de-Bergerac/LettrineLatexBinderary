
\documentclass[a4paper,12pt,german]{article}
\usepackage{iftex}
\iftutex
  \usepackage{fontspec}
  \setmainfont{erewhon}           % erewhon (Utopia)
\else
  \usepackage{fourier}            % fourier (Utopia) und
  \renewcommand{\ttdefault}{lmtt} % Latin Modern Typewriter fonts
\fi

\usepackage[a4paper,text={150mm,240mm},centering]{geometry}
\usepackage{microtype}

\usepackage{lettrine}
\usepackage{graphicx,color}
\usepackage{lettrine}

\usepackage{babel}

\newcommand{\MF}{{\small\scshape metafont}}
\newcommand{\MP}{{\small\scshape metapost}}
\setlength{\parindent}{0pt}
\sloppy
\begin{document}
\thispagestyle{empty}

\begin{center}
\large\bfseries Einige Beispiele zur Anwendung des lettrine-Pakets
\end{center}

\vspace{\baselineskip}
\textit{Einfachste Verwendung (2 Zeilen) :}\\
\verb+\lettrine{E}{in} erstes Beispiel...+

\lettrine{E}{in} erstes Beispiel zeigt das Standardverhalten von lettrine.
Es wird eine Initiale über zwei Zeilen produziert, gefolgt vom Text zwischen
den geschweiften Klammern, der als Kapitälchen gesetzt wird. Der weitere
Text umfliesst die Initiale.

\vspace{\baselineskip}
\textit{Lettrine auf einer Zeile (option {\ttfamily\upshape lines=1}) :}\\
\verb+\lettrine[lines=1]{E}{in} zweites Beispiel...+

\lettrine[lines=1]{E}{in} zweites Beispiel zeigt, wie eine Initiale auf
einer einzelnen Zeile aussieht. Auch hier ist der geklammerte Text in
Kapitälchen gesetzt.

\vspace{\baselineskip}
\textit{Initiale über drei Zeilen (option {\ttfamily\upshape lines=3}) :}\\
\verb+\lettrine[lines=3]{E}{in} drittes Beispiel...+

\lettrine[lines=3]{E}{in} drittes Beispiel in dem die Initiale über drei
Zeilen gesetzt wird. Beachten Sie die Einrückung der zweiten und dritten
Zeile. Diese kann mit dem Parameter \verb+nindent= + beeinflusst werden. Die
Einrückung der ersten Zeile wird hingegen mit dem Parameter \verb+findent= +
beeinflusst.

\vspace{\baselineskip}
\textit{Initiale vollständig im Randbereich} :\\
\verb+\lettrine[lhang=1,findent= .3em,nindent=0pt,lines=3]{V}{erschieben}+

\lettrine[lhang=1,findent= .3em,nindent=0pt,lines=3]{V}{erschieben}
wir nun im vierten Beispiel die Initiale in den Randbereich.
Dieses Verhalten wird durch den Parameter \verb+lhang= + gesteuert.

\vspace{\baselineskip}
\textit{Initiale, vergrössert und teilweise im Randbereich} :\\
\verb+\lettrine[lines=3, lhang=0.33, loversize=0.25]{A}{uch}+

\lettrine[lines=3, lhang=0.33, loversize=0.25]{A}{uch}
die Vergrösserung der Initiale ist möglich. Die Vergrösserung läuft
über die Variable \verb+loversize= + . Sie müssen das Ergebnis nicht
unbedingt schön finden. Es sieht nach meiner Meinung besser aus, wie das
vollständige Verschieben in den Randbereich.

\vspace{\baselineskip}
\textit{Setzen wir nun eine %französisches
         Anführungszeichen davor} :\\
\verb+\lettrine[ante=\frqq]{M}{it} dem Parameter ...+

\lettrine[ante=\frqq]{M}{it} dem Parameter \verb+ante= + kann auch ein
beliebiger Text vor die Initiale gesetzt werden. In der Praxis dürften
wohl nur Anführungszeichen dafür in Frage kommen.

\vspace{\baselineskip}
\verb+\def\lglqq{\raisebox{-\baselineskip}{\glqq}}+\\
\verb+\lettrine[ante=\lglqq]{M}{it} dem Parameter ...+
\def\lglqq{\raisebox{-\baselineskip}{\glqq}}

\lettrine[ante=\lglqq]{M}{it} dem Parameter \verb+ante= + kann auch ein
beliebiger Text vor die Initiale gesetzt werden. In der Praxis dürften
wohl nur Anführungszeichen dafür in Frage kommen.

\newpage
\textit{Wir verkleinern die Initiale nun um 10\% und heben sie
um 15\% wegen des {\glqq}Q{\grqq}}:\\
\verb+\lettrine[lines=4, loversize=-.15, lraise=.15]{Q}{ualität}+

\lettrine[lines=4, loversize=-.15, lraise=.15]{Q}{ualität} hat ihren
Preis.  Und wenn es nur die Zeit ist, um zu lernen wie Sie solche Spielereien
anstellen können. Bei den Ergebnissen lohnt sich aber die Mühe. Welche
Parameter diesmal was beeinflussen, lasse ich Sie nun selbst herausfinden.
Wie Sie sehen, ragt der Unterstrich des {\glqq}Q{\grqq} nicht in den Text
hinein.

\vspace{.5\baselineskip}
\textit{Andere Möglichkeit: Nochmal das {\glqq}Q{\grqq} in normaler
  Größe aber mit einer Zeile mehr darunter}
\verb+\lettrine[lines=4,depth=1]{Q}{ualität}+

\lettrine[lines=4,depth=1]{Q}{ualität} hat ihren Preis.  Und wenn es nur die
Zeit ist, um zu lernen wie Sie solche Spielereien anstellen können. Bei den
Ergebnissen lohnt sich aber die Mühe. Welche Parameter diesmal was
beeinflussen, lasse ich Sie nun selbst herausfinden. Wie Sie sehen, ragt der
Unterstrich des {\glqq}Q{\grqq} auch nicht in den Text hinein: eine Zeile mehr
wurde darunter mit \verb+depth=1+ frei gemacht.

\vspace{.5\baselineskip}
\textit{Verwendung der Option {\ttfamily\upshape slope}}, damit der Text
der Neigung des {\glqq}A{\grqq} folgt:\\
\verb+\lettrine[lines=4, slope=0.6em, findent=-1em,+\\
\verb+          nindent=0.6em]{\A}{uch}...+

\lettrine[lines=4, slope=0.6em, findent=-1em, nindent=0.6em]{A}{uch} eine
Neigung kann angegeben werden. Damit werden die Löcher neben geneigten
Buchstaben nicht so gross. Selbst eine negative Neigung ist möglich, damit
bietet auch das {\glqq}V{\grqq} keine Schwierigkeiten mehr. Wie das beim
{\glqq}V{\grqq} aussieht, sehen wir uns beim nächsten Beispiel an.

\vspace{.5\baselineskip}
\textit{Verwendung der Option {\ttfamily\upshape slope}, damit der Text
der Neigung des {\ttfamily\upshape V} folgt; Das {\ttfamily\upshape V} ragt
zusätzlich halb in den Rand hinein
(Option {\ttfamily\upshape lhang=0.5} :})\\
\verb+\lettrine[lines=4, slope=-0.5em, lhang=0.5, findent=.5em,nindent=0pt]+\\
\verb+         {V}{iel} ist...+

\lettrine[lines=4, slope=-0.5em, lhang=0.5, findent=.5em, nindent=0pt]{V}{iel}
ist hier nicht anders. Nur die negative Neigung und das Hereinragen in den
Rand. Ob Ihnen das Ergebnis gefällt müssen Sie selber entscheiden.
Sie sehen aber, das das {\glqq}V{\grqq} wirklich keine Schwierigkeiten bietet.
Der Unterschied zum vorhergehenden Beispiel ist nicht besonders gross.

\vspace{.5\baselineskip}
\textit{Ändern wir nun die Schriftfamilie für die Initiale
(hier AvantGarde bold italique):}\\
\verb+\renewcommand{\LettrineFontHook}{\fontfamily{pag}\fontencoding{T1}%+\\
\verb+                   \fontseries{bx}\fontshape{it}}+\\
\verb+\lettrine[findent=.3em]{A}{uch} ein Wechsel...+

{% (Ändern des lokalen fonts)
\renewcommand{\LettrineFontHook}{\fontfamily{pag}\fontencoding{T1}\fontseries{bx}\fontshape{it}}

\lettrine[findent=.3em]{A}{uch} ein Wechsel der Schriftfamilie ist problemlos
möglich. Hier verwenden wir Avantgarde und setzen mit der Option
\verb+findent= + den horizontalen Abstand des eingerückten Texts.
\par}

\vspace{.5\baselineskip}
\textit{Ändern wir nun die Schriftfamilie und die Farbe für die Initiale
(hier yfrak in Grau) :}\\
\verb+\renewcommand{\LettrineFontHook}{\fontfamily{yfrak}\fontencoding{T1}+\\
\verb+   \color[gray]{0.5}}\lettrine[loversize=0.1]{A}{uch}...+

{% (Aendern des lokalen fonts)
\renewcommand{\LettrineFontHook}{\fontfamily{yfrak}\fontencoding{T1}
    \color[gray]{0.5}}

\lettrine[loversize=0.1]{A}{uch} ein Wechsel der
Schriftfamilie ist problemlos möglich. Hier verwenden wir yfrak,
etwas vergrössert mit der Option \verb+loversize= +, und wir schreiben
die Initiale in Grau mit \verb+\color[gray]{0.5}+.
\par}

\newpage
\begin{center}
\large\bfseries Verwendung eines PostScript-Bildes als Initiale
\end{center}

\vspace{\baselineskip} Wenn die erwünschte Initiale nicht als Zeichen eines
Fonts, sondern als Bild im Postscript-Format vorliegt, kann ebenfalls
\verb+\lettrine+ verwendet werden. Es genügt,
die Boolsche Variable \texttt{image=true} zu benützen; z.B. so:

\vspace{.5\baselineskip}
{% Gruppierung, um die  LOKALEN Definitionen zu schützen
\fontfamily{yfrak}\fontencoding{T1}\selectfont\Large
\renewcommand{\LettrineTextFont}{\relax}
\lettrine[image=true, lines=3, lhang=.2, loversize=.25, %
          lraise=-.05, findent=0.1em, nindent=0em]
{W}{er} reitet so spät durch Nacht und Wind?\\
Es ist der Vater mit seinem Kind;\\
Er hat den Knaben wohl in dem Arm,\\
Er fa{\ss}t ihn sicher, er hält ihn warm.
\par}

\vspace{\baselineskip} Und hier der zum Beispiel gehörende \LaTeX{}--Code:
Das erste Argument von \verb+\lettrine+ war \verb+W+. Die Option \texttt{image=true}
lädt dann die Datei \verb+W.eps+. Das Suffix \verb+.eps+ kann -- dank des
Pakets \verb+graphicx.sty+ -- weggelassen werden.

\begin{verbatim}
{\fontfamily{yfrak}\fontencoding{T1}\selectfont\Large
\renewcommand{\LettrineTextFont}{\relax}
\lettrine[image=true, lines=3, lhang=.2, loversize=.25, %
          lraise=-.05, findent=0.1em, nindent=0em]
{W}{er} reitet so spät durch Nacht und Wind?
Es ist der Vater mit seinem Kind;
Er hat den Knaben wohl in dem Arm,
Er fa{\ss}t ihn sicher, er hält ihn warm.\par}
\end{verbatim}

Zur Darstellung dieses Beispiels müssen folgende Pakete installiert sein:
\begin{itemize}
\item \verb+graphicx.sty+,
\item die Schriften \verb+yfrak.pfb+ im type\,1-Format
   von Yannis~\textsc{Haralambous},
\item das Paket \verb+blacklettert1+ von Thorsten~\textsc{Bronger}.
\end{itemize}

Die gothische Initiale \glqq W\grqq{} in diesem Beispiel können Sie mit dem
Programm \MP{} aus den \MF{}-Sourcen und \verb+yinitW.mf+ erzeugen.

Falls Sie eine PDF-Datei erzeugen wollen, müssen Sie die Datei \verb+W.eps+
in eine PDF-Datei \verb+W.pdf+ umwandeln (mit Hilfe von \verb+epstopdf+).

\verb+\lettrine+ unterstützt die Verwendung der Formate:
\texttt{pdf}, \texttt{png}, \texttt{jpeg} oder \MP{} als Initiale.

\vfill
\begin{flushright}
  Deutsche Version Georg \textsc{Wagner}\\
  \texttt{g.wagner@datacomm.ch}\\
  Mai 2003, ergänzt September 2014
\end{flushright}

\end{document}

%%% Local Variables:
%%% mode: latex
%%% coding: utf-8
%%% TeX-master: t
%%% TeX-engine: xetex
%%% End:
